\documentclass[11pt]{article}
\usepackage{graphicx}
\usepackage{hyperref}
\usepackage{fancyhdr}
\usepackage{setspace}
\pagestyle{fancy}
\fancyhead{}
\fancyfoot{}
\fancyfoot[C]{-\thepage-}
\fancyfoot[L]{ID:997900158}
\fancyfoot[RO]{MPM406 - Random Clinical Trial}
\renewcommand{\footrulewidth}{0.4 pt}
\renewcommand{\headrulewidth}{0 pt}

\hypersetup{colorlinks = true, linkcolor = blue, citecolor = blue}
\title{Two Factorial Randomized Clinical Trial comparing dose and timing of estrogen supplementation on the prevention of athersclerotic cardiovascular disease in post menopausal women}
\author{Student ID: 997900158}
\date{\today}
\begin{document}
	\maketitle
	\begin{abstract}
		This two factorial multicenter randomized clinical trial was conducted on a convenience sample of peri-menopausal women presenting to North American Kaiser Permanente hospitals in 2001-2002.
		The women were given differing levels of estrogen supplementation both before and after the onset of menopause, and the level of carotid inter-media thickness measured to assess progression of athersclerotic cardiovascular disease, as well as cases of clinical athersclerotic cardiovascular disease recorded.
		The women were followed up for 8 years (until 2010) results revealed early supplementation of low dose estrogen is as protective as high dose, but with less side effects.

	\end{abstract}

\onehalfspace
	\section{Introduction} 
		Athersclerotic cardiovascular disease is the buildup of fatty deposits and white blood cells in arterial walls which results in hardening of the arteries which can cause thromboembolism and myocardial infarction.
		It is an important cause of cardiovascular disease in people but prevalence rates are difficult to determine due to its subclinical signs until a clinical athersclerotic cardiovascular disease event.
		Even though there has been an overall decline in the death rate due to cardiovascular disease in the United States over several decades, the rate of decline is less for women and especially African-American women \cite{Mosca1997}.
		In 1994, Cardiovascular disease killed a half million women and accounted for over 40\% of all deaths in women, more than all forms of cancer combined\cite{AHA1997}.
		While men are more commonly affected by cardiovascular disease, risk increases rapidly in women as they age, doubles every decade after age 55\cite{Gordon1978}. 
		This age related increase in risk is in part due to the onset of menopause (the absence of a menstrual cycle in the previous 12 months) which occurs at an average age of 51 but can range between 45 and 55 years \cite{Gold2012}. 
		It has been shown that reduced circulating estradiol during menopause increases atherogenic lipids and reduces carotid blood flow, causing increased incidence of athersclerotic cardiovascular disease \cite{Hodis}.
		Supplemental oestrogen has been used for some time to treat symptoms of menopause and its associated increase in athersclerotic cardiovascular disease risk, however serious side effects ( higher risk of breast cancer, increased blood clots and endometrial cancer) have been documented \cite{Gold2012}.
		Recent research has also investigated the cardiovascular protection associated with menopausal hormonal therapy and concluded no overall benefit of supplemental estrogen overall, and a possible interaction between time of therapy initiation and protection provided \cite{Anderson2004,Prentice2009}.
		Since the initial research questioning the efficacy of menopausal hormonal therapy was completed, there have been numerous innovations in the form and dose of estrogen deliverer have occurred, and it is hypothesised that these new forms/dosing regimes may provide improved protection.



		This two-by-two factorial randomized clinical trial is designed to assess the effects of initiating estradiol supplementation at different doses and different times around the onset of menopause on the risk of athersclerotic cardiovascular disease.
		The study Hypothesis is that initiating estrogen therapy before the onset of menopause is protective of athersclerotic cardiovascular disease, and that a reduced dose of estrogen provides the same protection with reduced side effects.


		\newpage
	\section{Methods} 
		The study population was a convenience sample drawn from women presenting to Kaiser Permanente health clinics between 2000-01-01 and 2000-12-31, aged between 40 and 50 that were undergoing signs of peri-menopause (last cycle less than 12 months ago, but no regular monthly cycle).
		Kaiser Permanente is a managed care consortium based in northern California. It has almost 15,000 physicians operating in 650 facilities spread over 9 states, and serves nearly 9 million members \cite{Rauber}.
		On identifying a possible study subject, a full physical examination and medical history was taken, and where possible validated with a central database of historical medical records maintained by Kaiser Permanente.
		During this physical examination BP, BMI, race, and age were measured, and lifestyle factors such as smoking and physical activity levels recorded.
		Blood was also collected during this initial examination and levels of lipoprotiens measured.


		Any woman with a previous hysterectomy, had taken estrogen supplements in the past, or had experienced any prolonged angina was disqualified from the study as hysterectomy causes instant menopause at any age, and angina may be a sign of preexisting heart conditions.
		The results of blood screening was assessed and any women with high levels of Low density lipoprotiens was excluded.
		The thickness of carotid artery intima-media was recorded  by ultrasound and any individual with stenosis greater than 75\% was excluded, as this is considered a cutoff for clinically significant disease \cite{Maseri2003}. 
		This initial measurement of carotid inter-media thickness also provided a baseline against which future measurements could be compared to to assess disease progression.
		Women who's medical history differed significantly from that recorded in Kaiser Permanente records was also excluded, as this may indicate future compliance and recall issues.
		

		Sample size requirements were calculated using a subclinical athersclerotic cardiovascular disease rate of 25\%, to detect a difference of 1.5 for 9 years of study with 80\% power and 95\% confidence.
		Adding 10\% attrition ( we observed 9\% ) the final sample size for each group was 4682. 
		Our sample size was much larger, allowing us to assess the interaction between timing and dose of intervention.


		Interim analysis were completed at bi-yearly intervals by an independent data and safety monitoring committee, and the decision to proceed based on participant safety (no increased rates of side effects or athersclerotic cardiovascular disease) and the perceived benefit of continuing with study.
		O’Brien-Flemming spending strategy was used to generate stopping boundaries for each planned analysis.


		Two factorial design was chosen to allow simultaneous comparison of dose and timing effects, and the use of a very large sample size allowed the assessment of interactions at sufficient power. 
		The two dose levels chosen were 0.625mg and 0.3mg of conjugated equine estrogens, delivered per os (hence denoted high and low respectively).
		0.625 mg SID PO is the standard therapy currently in use, the lower dose of 0.3mg has been suggested as a way to maximise benefit ( menopause symptoms, CVD,osteoporosis), while minimising harmful side effects (cancer risk, blood clots)  associated with estrogen.
		The two timing levels chosen were initiation of therapy during peri menopause (early intervention), or initiation 3 years after menopause (late intervention).
		The use of a two factorial design resulted in 4 study groups, as described in \hyperref[Figure 1]{figure1}.


		Patients were randomised using a centralised allocation procedure, with both patient and physician blinded to allocation method. 
		There was no blocking or stratification of study participants performed.
		After eligibility was established, the physician sent an auto generated email containing patient information to a server maintained in the researchers office at Kaiser Permanente's headquarters in Oakland, California. 
		This computer used a schedule of random numbers generated from open radio frequency atmospheric noise \cite{Eddelbuettel2009}  and study participants were allocated to one of the four groups based on this number.
		The allocation, general patient details, and source clinic and timestamp of the allocation request were recorded and compared to doctor's own records of assignment to ensure accuracy.
		The result of this allocation was returned to the physician in an email with a number 1-4 and the appropriate treatment initiated.


		Following assignment, all patients received a 6 month supply of pills, regardless of group assignment.
		This pill pack was in nondescript packaging, with only a number label, and was renewed every 6 months by mail from Kaiser Permanente headquarters.
		Even after assignment, the allocation sequence and allocation itself was concealed from study participant and physician.
		Those that were assigned late delivery of either dose were initially given sugar pills that had same appearance, taste, and smell as the real estrogen pills. 
		3 years after menopause was determined to have taken place ( more than 12 months amenorrhoea) these study participants had their next 6 monthly supply of pills converted to estrogen, at a dose depending on their assignment. Again there was no way to tell the difference between pills having different doses.

		
		Patients were followed for 9 years (2001-01-01 - 2009-12-31) and had annual checkups with their Kaiser Permanente physician during the study period.
		During this examination, a standard examination and history was performed, as well specific questions relating to the onset of menopause, symptoms experienced during the year, and any incidence of angina or clinical cardiovascular disease.
		The incidence of side effects of estrogen therapy monitored included cancer (breast, colon, endometrial), and clotting events (Deep Vein Thrombosis, Ischaemic stroke) was also asked and recorded.
		All women had their responses checked against actual medical history, improving accuracy of data and minimising recall bias.


		To assess the progression of subclinical athersclerotic cardiovascular disease, women had a carotid ultrasound to measure carotid artery intima-media thickness.
		The thickness of carotid artery intima-media was recorded and compared to previous measurements to assess disease progression.
		This intervention was well tolerated as women were presenting to their health professional for an annual checkup anyway, and the Carotid ultrasound is quick and non-invasive
		In addition, the quality of these measurements were maintained by using a trained ultrasonographer at each of the Kaiser Permanente facilities.
		Before the beginning of enrollment, these ultrasonographers were trained at a central location on a standard protocol for taking measurements, and the inter-observer agreement assessed by each ultrasonographer completing an examination of the same 5 test subjects.


		All patients were thoroughly briefed on the study design and interventions, and signed informed consent and liability release forms.


	\subsection{Statistical Evaluation}
		The primary study endpoint was progression of subclinical athersclerotic cardiovascular disease defined as an increase in carotid inter-media thickness of more than 0.0035mm per year diagnosed on carotid ultrasound.
		A second study endpoint was clinical athersclerotic cardiovascular disease, which signs included stroke, congestive heart disease, and death due to cardiac causes.


		Kaplan-Meier survival curves for overall survival were compared by log-rank test, and the unadjusted HR (and 95\% CI) was calculated using a Cox regression analysis.
		As a secondary analysis, potential covariates (race, age, smoking, physical activity levels) were included in the Cox regression model to generate an adjusted HR for overall survival.



		All analysis was completed on an intention to treat basis, and all statistical analysis were completed in R \cite{RCoreTeam2012} on data that had a random hashing algorithm performed before analysis, ensuring blinding of statisticians.



	\section{Results}
		43,426 women were initially enrolled in this study, with 42,236 assigned to a treatment group and 38,234 completing and were analysed in the study
		The number of people allocated to each group and the number missing from each group can be seen in  \hyperref[flow]{Study flow}
		Approximately 900 (9\%)  participants were missing from each group which seems high but is to be expected on such a large study spread over multiple locations. 
		The losses from each group were equal (non differential dropout), and reasons ranged from changed address, loss of contact, or death due to other non cardiac causes)


		Baseline characteristics of study participant can be seen in \hyperref[table1]{Table 1}.


		Table 3 (no data required) shows the incidence of increased carotid inter-media thickness in all groups, and also adjusted Hazard ratios associated with included covariates.
		As can be seen, women who took the low dose and started during perimenopause had a lower risk of developing athersclerotic cardiovascular disease, while.
		Subgroup analysis was performed on African American women, and smoking women.
		African-American women have a higher incidence of athersclerotic cardiovascular disease than others, and our data reflected this, but the reduced risk of athersclerotic cardiovascular disease with low dose early start estrogen was the same for African American women as the whole population.
		Women who smoke are also at higher incidence of athersclerotic cardiovascular disease, but once again our findings held for smokers and non smokers alike.


		The incidence of side effects that can be caused by estrogen is shown in \hyperref[table2]{Table 2}, Comparison rates in post menopausal women not undergoing menopausal hormonal therapy are shown for comparison.
		As can be seen from Table 2, there is no increased risk of side effects associated with menopausal hormonal therapy.


		The compliance rate among study participants was assessed by measuring estrogen levels in blood of a sample (n=100) of participants receiving estrogen at least 2 years following the onset of menopause. Compliance was found to be good, at 93\%, and no adjustment was made for those found to be non compliant.

		
		The rates of diagnosis on each ultrasonographer were compared to ensure inter-observer agreement.
		As each ultrasonographer completed many measurements, each individual observer distribution was compared to the average distribution of all observers, to determine if any one observer was over or under measuring carotid inter-media thickness.
		The interobserver correlation was above 90\% and no single observer consistently over or under measured carotid inter-media thickness.


	\section{Discussion} 


	\subsection{Strengths and Limitations}
		Data quality was a strength of this study.
		Cross validating the patients oral medical histories with the actual recorded histories from the Kaiser Permanente central databases ensures accuracy and reduces recall bias.
		This improved the accuracy of measurement of secondary outcome, clinical athersclerotic cardiovascular disease events.
		Study group allocation was entirely random and repeatable, and the comparison of allocation and actual treatment records allowed analysis to be undertaken on an accurate intention to treat basis. 
		The effectiveness of this random allocation procedure can be seen in the comparability of groups in \hyperref{table1}, and shows the study has good internal validity.
		Primary outcome results were improved by the use of trained ultrasonographers who had been assessed pre trial (see above)


		The triple blinding methods used in this study were also a strength, with study participants, consulting physicians, and statisticians all unaware of which group participants/data were assigned to.
		There was no need to assess the efficacy of blinding as any unmasking could only occur from physical changes in the participant, over which we had no control. 


		A potential limitation of this study was the selection of study participants from Kaiser Permanente hospitals.
		Rates of insurance are correlated with Socio economic status and Education level, both factors that are also associated with use of menopausal hormonal therapy.
		There may be some relationship between lifestyle factors/SES and progression of athersclerotic cardiovascular disease, and our attempt to control for these factors (correcting for smoking and activity levels) may be insufficient.
		The use of a convenience sample from Kaiser Permanente external validity may be limited, however it is likely that Kaiser Permanente clientele represents a good sample of middle and upper class women hence findings are valid across this population.


		Another potential limitation is the loss of blinding to study participants based on the effects of treatment. Some women experience bleeding when taking an estrogen supplement, and this would remove blinding.


		The use of carotid inter-media thickness measurements to diagnose significant athersclerotic cardiovascular disease may be inappropriate as it has been shown the presence of 'vulnerable plaques' vs a single artery wall thickness measurement is more predictive of a clinical athersclerotic cardiovascular disease event \cite{Maseri2003}.
		

	\section{Conclusion}
		This study demonstrated a benefit from low dose estrogen therapy started early in peri-menopause in protecting against subclinical and clinical athersclerotic cardiovascular disease.
		Side effect rates in this study reflected those reported elsewhere, however recent work has shown alternative agents may have reduced side effect rates.
		The use of early delivery of alternative estrogen preparations and delivery routes is one area of research where further investigation is warranted.


	\section{Figures}

\begin{figure*}[h!]
	\centering
	\includegraphics[scale=0.5]{figure1.jpg}
	\caption{Flow Diagram showing study participant allocation}
	\label{flow}
\end{figure*}

\begin{figure*}[h!]
	\centering
	\includegraphics[scale=0.6]{table1.jpg}
	\caption{Characteristics of study participants.}
	\label{table1}
\end{figure*}
 
\begin{figure*}[h!]
	\centering
	\includegraphics[scale=0.7]{table2.jpg}
	\caption{Incidence of Side effects in study groups}
	\label{table2}
\end{figure*}

\clearpage
\bibliographystyle{unsrt}
\bibliography{wom}



\end{document}

%%%%%

% interactoins - 
%% make streamlined as possible. but need at least one confounder, not necc any interactions, can say did not find any (kim paper for technique - none stat sig so not included). potential effect modify - wualitative. 
% if include need to show OR with and without interactions.
% table 1 - cats in study,. throw in other factors that wont include - make them the same. e.g. age w exposure outcome and not on causal path. - no sig p value. 
% fig 1 - exposure b \emph{Bartonella sp.} , outcome ev.

% diagram bartonella causing uveitis, age associated w bartonella and uveitis (double ended arrow.) then show in table.( anythign in table that same do not need to have in diagram.
